\documentclass[11pt,a4paper]{article}
\usepackage{times}
\usepackage{graphicx}
\usepackage[
backend=biber,
style=numeric,
]{biblatex}

\addbibresource{ref.bib}

\title{Testing and Programmability of a Graph Framework}


% Questions:
% Best run or avg of many runs? Results and Analysis suggests best run, but Methodology section suggests average.
% Source?
% PR - one iteration?
% Which datasets to show in paper?

\begin{document}
\maketitle

\section{Introduction}
\section{Testing of a Graph Framework}
\subsection{Metrics}
Discuss different metrics and their importance:
\begin{itemize}
    \item Runtime
    \item Nodes / Edges Visited
    \item \textbf{MTEPS}
    \item Search depth
\end{itemize}
\subsection{Methods}
How we went about collecting these metrics:
\begin{itemize}
\item Ability to turn performance evaluation on and off
    \begin{itemize}
    \item Conditional outside CUDA code
    \end{itemize}
\item Collect runtime with performance evaluation turned off
\item How to track nodes/edges visited varies by primitive
    \begin{itemize}
        \item BFS/SSSP - increment a counter each time an edge is visited
        \item BC - use frontier
        \item PR - constant
    \end{itemize}
\item Multiple runs / sources
\item Throw out anomalous results
\item Other items we track for reproducibility  
    \begin{itemize}
        \item GPU info
        \item System info
        \item Time
        \item Git commit
        \item Etc.
    \end{itemize}
\end{itemize}
\subsection{Comparison}
Why we compare to Gunrock:
\begin{itemize}
    \item Previously shown to be among the best-in-class \cite{Wang:2017:GGG}
\end{itemize}
\subsection{Results}
Show and discuss results:
\begin{itemize}
    \item Runtime - Essentials vs Gunrock 1.0+ - V100 - various datasets - all primitives
    \item MTEPS vs Number of Edges - Essentials - A100 - various datasets - all primitives
    \begin{itemize}
        \item Show how well the framework scales
    \end{itemize}
\end{itemize}
\section{Programmability of a Graph Framework}
\section{Conclusion}
\printbibliography

\end{document}
